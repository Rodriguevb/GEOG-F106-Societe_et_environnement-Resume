\section{Questions et réponses}
\subsection{Pattyn}
\subsubsection{Logique}
\begin{description}
	\item [Le réchauffement est plus fort en antarctique qu'en arctique]
    \color{cyan}\color{black}
    
	\item [Révolution Néolithique est le passage de chasse/pêche à l'agriculture]
    \color{cyan}Vrai\color{black}
   
	\item [Après l'ère Néolithique vient l'ère Anthropique]
    \color{cyan}\color{black}
\end{description}

\subsubsection{Pourquoi il y a un réchauffement important en Antarctique ?}
\subsubsection{Classer les gazs à effet de serre selon leur importance (quantité ou forcage radioactif?)}
\subsubsection{Cycle de formation de l'ozone à compléter}
\subsubsection{Pourquoi il y a plus d'ozone troposphérique en milieu rural qu'en zone urbaine ?}
\subsubsection{Pourquoi le réchauffement est plus élevé au pole ?}



\subsection{Decroly}
\subsubsection{Logique}
\begin{description}
	\item [Population sur Terre en l'an]
    \color{cyan}0\color{black}
	
    \item [Lula a voulu diminuer les productions de soja.]
    \color{cyan}Faux\color{black}
    
	\item [Nord Sahel, extension de la culture pluviale au détriment du pastoralisme]
    \color{cyan}\color{black}
    
	\item [question sur les finalités de l'agriculture au Sahel avant la colonisation]
    \color{cyan}\color{black}
	
    \item [Augmentation des prix agricoles au Niger entre 1950 et 1970]
    \color{cyan}\color{black}
    
	\item [Pourcentage d'ea douce amenée sur terre par l'évaporation des océans]
    \color{cyan}\color{black}
    
	\item [Projet de transfert d'eau en Tunisie au bénéfice du Tourisme]
    \color{cyan}Vrai\color{black}
    
	\item [GAP financé par les organismes d'aide au développement]
    \color{cyan}Faux\color{black}
    
	\item [GAP accompagné d'une réforme agraire pour les Kurdes]
    \color{cyan}Faux\color{black}
    
	\item [La Turquie considère eaux du T et E comme ressource naturelle nationale]
    \color{cyan}Vrai\color{black}
    
	\item [Reduction de la biodiversité plus rapide que rejets Pb]
    \color{cyan}Faux\color{black}
    
	\item [Population 25.000.000 en l'an 0]
    \color{cyan}Faux\color{black}
    
	\item [Judéo-Chrétien n'a pas de valeurs sacrées aux éléments de la nature]
    \color{cyan}Vrai\color{black}
    
	\item [Modernisme, représentation externe de la nature]
    \color{cyan}\color{black}
    
	\item [Fin 18ème, athoropocène]
    \color{cyan}Vrai\color{black}
    
	\item [Dès 1910, la production de caoutchouc hollandaise a supplanté la production brsilienne]
    \color{cyan}Vrai\color{black}
    
	\item [La rouille su-américaine attaque plus les hévéas cultivés que les naturels]
    \color{cyan}Vrai\color{black}
\end{description}

\subsubsection{Définition : obsolescence}
\color{cyan}
\begin{description}
	\item [L'obsolescence] est le fait pour un produit d’être dépassé, et donc de perdre une partie de sa valeur en raison de la seule évolution technique (on parle alors d'\textbf{obsolescence technique}) ou de la mode (on utilise alors plutôt le mot "démodé") ;
	\item [L’obsolescence programmée] est le nom donné à l'ensemble des techniques visant à réduire la durée de vie ou d'utilisation d'un produit afin d'en augmenter le taux de remplacement.\color{black}
\end{description}
\color{black}

\subsubsection{Pourquoi les USA ont tellement de terres irriguées ?}
\color{cyan}
\color{black}

\subsubsection{Pourcentage d'agricultue en Israêl ? Quel est le secteur le plus exploité ?}
\color{cyan}
\color{black}

\subsubsection{Dans quel domaine la Turquie met toute son économie ?}
\color{cyan}
Dans le tourisme
\color{black}

\subsubsection{Question sur le pourcentage de forêt amazoniene déboisée}
\color{cyan}
\color{black}

\subsubsection{Question sur le pourcentage d'unités de conservation de la forêt brésilienne}
\color{cyan}
\color{black}

\subsubsection{Quelles sont les 3 motivations principales du développement de l'Amazonie et pourquoi il n'y a pas de réforme agraire ?}
\color{cyan}
\color{black}

\subsubsection{Pourquoi le Brésil a encouragé la déforestation et pourquoi il n'y a pas de réforme agraire}
\color{cyan}
Entre 1970 et 1990, la logique était un déboisement motivé par la spéculation foncière ; On achète des terres pour faire augmenter leur prix et ensuite on les revends. Dans les années '90, le déboisement était plutôt motivé par la rentabilité économique des productions de soja et de viande de boeuf ; On a besoin de place pour ces productions car la demande mondiale est élevée.
\color{black}

\subsubsection{Pourquoi le Brésil préfère une réforme agraire ?}
\color{cyan}
Coexistence de 2 partis : oligarchie foncière et bourgeoise cafétière
Les facteurs influençant ...
\begin{itemize}
\item motifs socioplitiques ;
\item  motifs économiques ;
\item motifs stratégiques.
\end{itemize}
\color{black}